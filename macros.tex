% \usepackage{amsmath}
% \usepackage{amsthm}
% \usepackage{amsfonts}
% \usepackage{thmtools,thm-restate}
% \usepackage{bbm}
% \usepackage{mathtools}
\usepackage{natbib}

\usepackage{subcaption}
\usepackage[ruled,vlined]{algorithm2e}
\usepackage{caption}
\usepackage{graphicx}
% \usepackage{subfigure}
\usepackage{amsmath}
\usepackage{amsthm}
\usepackage{amsfonts}
\usepackage{thmtools,thm-restate}
\usepackage{bbm}
\usepackage{mathtools}
\usepackage{authblk}

% \usepackage[hidelinks]{hyperref}
% \usepackage[square,sort,comma,numbers]{natbib}
% \newtheorem{lemma}{Lemma}
% \newtheorem{theorem}{Theorem}
% \newtheorem{definition}{Definition}
% \newtheorem{sublemma}{Lemma}[lemma]
% \newtheorem{corollary}{Corollary}
\newcommand{\indep}{\perp \!\!\! \perp}
% \usepackage[backend=biber]{biblatex}
% \addbibresource{references.bib}
% \PassOptionsToPackage{numbers}{natbib}
% \setcitestyle{numerical, square}
% \setcitestyle{authoryear,open={(},close={)}}
\usepackage{capt-of}
% \usepackage{enumerate}
% \hypersetup{colorlinks,citecolor=blue}
\usepackage{comment}
% \newcommand{\faaiz}[1]{\textcolor{red}{{\bf Faaiz:} #1}}
\newcommand{\faaiz}[1]{#1}
% \newcommand{\R}[1]{\textcolor{red}{#1}}
\newcommand{\R}{\mathbb{R}}

\newcommand{\jef}[1]{\textcolor{blue}{{\bf Jef:} #1}}
\usepackage{amsmath}
\DeclareMathOperator*{\argmax}{arg\,max}
\DeclareMathOperator*{\argmin}{arg\,min}
\usepackage{comment}
\usepackage{microtype}
\usepackage{booktabs}
\usepackage{centernot}


%%%%%%%%%%%%%%% colors

\usepackage[many]{tcolorbox}
\usepackage{xcolor}
\usepackage{color}
\usepackage{colortbl}

%
\definecolor{MyLightGray}{gray}{0.925}
\definecolor{MyDarkGray}{gray}{0.55}
\definecolor{myblue}{rgb}{0.00, 0.45, 0.70}
\definecolor{mygreen}{rgb}{0.01, 0.62, 0.45}
\definecolor{myred}{rgb}{0.84, 0.37, 0.00}

%
%
\PassOptionsToPackage{hyphens}{url}
% \usepackage[hidelinks]{hyperref}
\usepackage{hyperref}
%
%
%
%
%
%

%
\definecolor{blue}{RGB}{0,114,178}
\definecolor{red}{RGB}{213,80,20}
%
\definecolor{green}{RGB}{0,158,115}
\definecolor{purple}{RGB}{204,121,167}
\definecolor{orange}{RGB}{230,159, 0}
\definecolor{pink}{RGB}{204,121,167}

\definecolor{ourmethod}{gray}{0.93}


%
\definecolor{sns0}{HTML}{1f77b4}
\definecolor{sns1}{HTML}{ff7f0e}
\definecolor{sns2}{HTML}{2ca02c}
\definecolor{sns3}{HTML}{d62728}
\definecolor{sns4}{HTML}{9467bd}
\definecolor{sns5}{HTML}{8c564b}

%
%
%
%
%
%
%
%
%


%
\definecolor{sns-orange}{HTML}{ff7f0e}
\definecolor{sns-ambiguous}{HTML}{00528C}
\definecolor{sns-nonambiguous}{HTML}{2CA9FF}
\definecolor{sns-blue}{HTML}{1f77b4}
%

%

%
\definecolor{solarized@base03}{HTML}{002B36}
\definecolor{solarized@base02}{HTML}{073642}
\definecolor{solarized@base01}{HTML}{586e75}
\definecolor{solarized@base00}{HTML}{657b83}
\definecolor{solarized@base0}{HTML}{839496}
\definecolor{solarized@base1}{HTML}{93a1a1}
\definecolor{solarized@base2}{HTML}{EEE8D5}
\definecolor{solarized@base3}{HTML}{FDF6E3}
\definecolor{solarized@yellow}{HTML}{B58900}
\definecolor{solarized@orange}{HTML}{CB4B16}
\definecolor{solarized@red}{HTML}{DC322F}
\definecolor{solarized@magenta}{HTML}{D33682}
\definecolor{solarized@violet}{HTML}{6C71C4}
\definecolor{solarized@blue}{HTML}{268BD2}
\definecolor{solarized@cyan}{HTML}{2AA198}
\definecolor{solarized@green}{HTML}{859900}



\theoremstyle{plain}

% Redefine the theorem environment to be enclosed in mainresult
\newtcolorbox{mainresult}{colback=solarized@violet!5!white,
colframe=solarized@violet,parbox, left=0.5mm, right=0.5mm,top=0.5mm,bottom=0.5mm}

% \newtcolorbox{mainresult}[1][]{colback=solarized@violet!5!white,
% colframe=solarized@violet, parbox=true, left=0.5mm, right=0.5mm, 
% top=0.5mm, bottom=0.5mm, breakable, title={#1}}

% Define the mainresult environment
\newtcolorbox{mainresultwithtitle}[2][]{colback=solarized@violet!7!white,
colframe=solarized@violet!7, 
colbacktitle=solarized@violet,
parbox=false, 
left=0.5mm, 
right=0.5mm, 
top=0.5mm, 
bottom=0.5mm, 
% breakable, 
title={#2}, 
#1}

% \newtcolorbox{importantresult}{breakable,colback=solarized@yellow!5!white,
% colframe=solarized@yellow,parbox, left=0.5mm, right=0.5mm,top=0.5mm,bottom=0.5mm}

% \newtcolorbox{importantresult}{
%   breakable,
%   colback=solarized@yellow!5!white,
%   colframe=solarized@yellow,
%   parbox=false,
%   left=0.5mm,
%   right=0.5mm,
%   top=0.5mm,
%   bottom=0.5mm,
%   % before upper={\setlength{\parskip}{1em}},
%   before={\setlength{\parskip}{1em}},
%   before upper={\vspace{-2.1em}}, % Remove the vertical space before the content
% % before upper={\vspace{-\baselineskip}}, % Remove the vertical space before the content
% }
\newtcolorbox{importantresultwithtitle}[2][]{
  % breakable,
  colback=solarized@cyan!7!white,
  colframe=solarized@cyan!7,
  colbacktitle=solarized@cyan,
  parbox=false,
  left=0.5mm,
  right=0.5mm,
  top=0.5mm,
  bottom=0.5mm,
  title={#2}, 
  #1
}


\newtcolorbox{whiteresult}{colback=solarized@violet!2!white,
colframe=solarized@violet,parbox, left=0.5mm, right=0.5mm,top=0.5mm,bottom=0.5mm}


\newtheorem{theorem}{Theorem}[section]
% Save the original theorem environment
\let\oldtheorem\theorem

% \renewenvironment{theorem}[1][\unskip]
%   {\begin{mainresult}\ifthenelse{\equal{#1}{\unskip}}{\oldtheorem}{\oldtheorem[#1]}}
%   {\end{mainresult}}


% \newcounter{theorem}[section]
% \renewcommand{\thetheorem}{\thesection.\arabic{theorem}}

% Define a new environment for assumptions within mainresult
\renewenvironment{theorem}[1][]
 {\refstepcounter{theorem}
  \begin{mainresultwithtitle}[title={Theorem \thetheorem\ifx#1\empty\else~(#1)\fi}]\noindent}
 {\end{mainresultwithtitle}}


% \renewenvironment{theorem}[1][\unskip]{%
%   \begin{mainresult}
%   \ifx#1\unskip\oldtheorem\else\oldtheorem[#1]\fi%
% }{%
%   % \endoldtheorem
%   \end{mainresult}
% }

\newcounter{proposition}[section]
\renewcommand{\theproposition}{\thesection.\arabic{proposition}}

% Define a new environment for assumptions within mainresult
\newenvironment{proposition}[1][]
 {\refstepcounter{proposition}
  \begin{mainresultwithtitle}[title={Proposition \theproposition\ifx#1\empty\else~(#1)\fi}]\noindent}
 {\end{mainresultwithtitle}}

% \newtheorem{proposition}[theorem]{Proposition}
% \let\oldproposition\proposition
% \renewenvironment{proposition}[1][\unskip]{%
%   \begin{mainresult}
%   \ifx#1\unskip\oldproposition\else\oldproposition[#1]\fi%
% }{%
%   % \endoldtheorem
%   \end{mainresult}
% }


\newcounter{lemma}[section]
\renewcommand{\thelemma}{\thesection.\arabic{lemma}}

% Define a new environment for assumptions within mainresult
\newenvironment{lemma}[1][]
 {\refstepcounter{lemma}
  \begin{mainresultwithtitle}[title={Lemma \thelemma\ifx#1\empty\else~(#1)\fi}]\noindent}
 {\end{mainresultwithtitle}}

% \newtheorem{lemma}[theorem]{Lemma}
% \let\oldlemma\lemma
% \renewenvironment{lemma}[1][\unskip]{%
%   \begin{mainresult}
%   \ifx#1\unskip\oldlemma\else\oldlemma[#1]\fi%
% }{%
%   \end{mainresult}
% }


\newcounter{corollary}[section]
\renewcommand{\thecorollary}{\thesection.\arabic{corollary}}

% Define a new environment for assumptions within mainresult
\newenvironment{corollary}[1][]
 {\refstepcounter{corollary}
  \begin{mainresultwithtitle}[title={Corollary \thecorollary\ifx#1\empty\else~(#1)\fi}]\noindent}
 {\end{mainresultwithtitle}}

 
% \newtheorem{corollary}[theorem]{Corollary}
% \let\oldcorollary\corollary
% \renewenvironment{corollary}[1][\unskip]{%
%   \begin{mainresult}
%   \ifx#1\unskip\oldcorollary\else\oldcorollary[#1]\fi%
% }{%
%   \end{mainresult}
% }

\newcounter{definition}[section]
\renewcommand{\thedefinition}{\thesection.\arabic{definition}}

% Define a new environment for assumptions within mainresult
\newenvironment{definition}[1][]
 {\refstepcounter{definition}
  \begin{importantresultwithtitle}[title={Definition \thedefinition\ifx#1\empty\else~(#1)\fi}]\noindent}
 {\end{importantresultwithtitle}}

% \theoremstyle{definition}
% \newtheorem{definition}[theorem]{Definition}
% \let\olddefinition\definition
% \renewenvironment{definition}[1][\unskip]{%
%   \begin{importantresult}
%   \ifx#1\unskip\olddefinition\else\olddefinition[#1]\fi%
% }{%
%   \end{importantresult}
% }

\newtheorem{assumption}[theorem]{Assumption}
% \let\oldassumption\assumption
% \renewenvironment{assumption}[1][\unskip]{%
%   \begin{mainresult}%
%   \ifx#1\unskip\oldassumption\else\oldassumption[#1]\fi%
% }{%
%   \end{mainresult}%
% }

% \newcounter{assumption}[section]
% \renewcommand{\theassumption}{\thesection.\arabic{assumption}}

% Define a new environment for assumptions within mainresult
% \newenvironment{assumption}[1][]
%  {\refstepcounter{assumption}
%   \begin{mainresultwithtitle}[title={Assumption \theassumption\ifx#1\empty\else~(#1)\fi}]\noindent}
%  {\end{mainresultwithtitle}}


 \newcounter{limitation}[section]
\renewcommand{\thelimitation}{\thesection.\arabic{limitation}}

% Define a new environment for assumptions within mainresult
\newenvironment{limitation}[1][]
 {\refstepcounter{limitation}
  \begin{mainresultwithtitle}[title={Limitation \thelimitation\ifx#1\empty\else~(#1)\fi}]\noindent}
 {\end{mainresultwithtitle}}

% \newtheorem{proposition}{Proposition}
% \theoremstyle{definition}
\newcounter{remark}[section]
\renewcommand{\theremark}{\thesection.\arabic{remark}}

% Define a new environment for assumptions within mainresult
\newenvironment{remark}[1][]
 {\refstepcounter{remark}
  \begin{importantresultwithtitle}[title={Remark \theremark\ifx#1\empty\else~(#1)\fi}]\noindent}
 {\end{importantresultwithtitle}}


% \newtheorem{remark}[theorem]{Remark}
% \let\oldremark\remark
% \renewenvironment{remark}[1][\unskip]{%
%   \begin{importantresult}
%   \ifx#1\unskip\oldremark\else\oldremark[#1]\fi%
% }{%
%   \end{importantresult}
% }

\newcommand{\red}[1]{\textcolor{red}{#1}}
\newcommand{\p}[0]{\mathbb{P}}
\newcommand{\E}[0]{\mathbb{E}}
\newcommand{\tarprob}[0]{\mathbb{P}_{(X,Y)\sim P^{\pi^*}_{X,Y}}}
\newcommand{\ttar}[0]{\mathbb{P}_{(X,Y)\sim \tilde{P}^{\pi^*}_{X,Y}}}
\newcommand{\behprob}[0]{\mathbb{P}_{(X,Y)\sim P^{\pi^b}_{X,Y}}}
\newcommand{\expb}[0]{\mathbb{E}_{(X,Y)\sim P^{\pi^b}_{X,Y}}}
\newcommand{\expt}[0]{\mathbb{E}_{(X,Y)\sim P^{\pi^*}_{X,Y}}}
\newcommand{\exptt}[0]{\mathbb{E}_{(\tilde{X},\tilde{Y})\sim P^{\pi^*}_{X,Y}}}
\newcommand{\expat}[0]{\mathbb{E}_{(X,Y)\sim \tilde{P}^{\pi^*}_{X,Y}}}
\newcommand{\expatt}[0]{\mathbb{E}_{(\tilde{X},\tilde{Y})\sim \tilde{P}^{\pi^*}_{X,Y}}}

\newcommand{\behcal}[0]{\mathbb{P}_{(X_i,Y_i)\sim P^{\pi^b}_{X,Y}}}
\newcommand{\expbcal}[0]{\mathbb{E}_{(X_i, Y_i) \sim P^{\pi^b}_{X,Y}}}


\newcommand{\todo}[1]{\textcolor{red}{#1}}
\newcommand{\ind}{\mathbbm{1}}
\newcommand{\Aspace}{\mathcal{A}}
\newcommand{\Xspace}{\mathcal{X}}



%%%%%%%%%%%%%% DT %%%%%%%%%%%%%%%%%



\newcommand{\ceil}[1]{\left\lceil#1\right\rceil}
\newcommand{\floor}[1]{\left\lfloor#1\right\rfloor}
\newcommand{\set}[1]{\left\{#1\right\}}
\newcommand{\abs}[1]{|#1|}
\newcommand{\norm}[1]{\left\lVert#1\right\rVert}
\DeclareMathOperator*{\supp}{supp}

\newcommand{\Var}{\mathrm{Var}}
\newcommand{\Prob}{\mathbb{P}}
\newcommand{\iid}{\overset{\mathrm{iid}}{\sim}}
\newcommand{\eqas}{\overset{\mathrm{a.s.}}{=}}
\newcommand{\Law}{\mathrm{Law}}
\newcommand{\io}{\mathrm{i.o.}}
\newcommand{\ev}{\mathrm{ev.}}
\newcommand{\eqd}{\overset{\mathrm{d}}{=}}
\newcommand{\neqd}{\overset{\mathrm{d}}{\neq}}

\newcommand{\Cauchy}{\mathrm{Cauchy}}
\newcommand{\Normal}{\mathrm{Normal}}
\newcommand{\Bernoulli}{\mathrm{Bernoulli}}
\newcommand{\Uniform}{\mathrm{Uniform}}
\newcommand{\Dirac}{\mathrm{Dirac}}

\DeclarePairedDelimiterX{\infdivx}[2]{(}{)}{#1\;\delimsize\|\;#2}
\newcommand{\KL}{D_{\mathrm{KL}}\infdivx*}
\newcommand{\JS}{\mathrm{JS}\infdivx*}
\newcommand{\fdiv}{\mathrm{D}_f\infdivx*}

\newcommand\ci{\perp\!\!\!\perp}

\DeclareMathOperator*{\ITE}{ITE}

\newcommand{\twinsupscript}{\mathsf{twin}}

\newcommand{\dee}{\mathrm{d}}

\newcommand{\Xspacedim}{d}
\newcommand{\A}{A}
\newcommand{\X}{X}
\newcommand{\xx}{x}
\newcommand{\Xt}{\widehat{\X}}
\newcommand{\At}{\A^\twinsupscript}
\newcommand{\ax}{a}

\newcommand{\Hspace}{\mathcal{H}}
\newcommand{\Hx}{H}
\newcommand{\hx}{h}

\newcommand{\Sx}{S}
\newcommand{\St}{\Sx^\twinsupscript}
\newcommand{\Sspace}{\mathcal{S}}
\newcommand{\Stspace}{\Sspace^\twinsupscript}
\newcommand{\sx}{s}

\newcommand{\Espace}{\mathcal{E}}
\newcommand{\Etspace}{\Espace^\twinsupscript}
\newcommand{\Eps}{\epsilon}
\newcommand{\Epst}{\Eps^\twinsupscript}
\newcommand{\ex}{e}
\newcommand{\agtsupscript}{\mathsf{agent}}
\newcommand{\Easpace}{\Espace^\agtsupscript}
\newcommand{\Epsa}{\Eps^\agtsupscript}
\newcommand{\epsa}{\ex^\agtsupscript}

\newcommand{\pix}{\pi}


\newcommand{\nx}{n}
\newcommand{\ix}{i}

\newcommand{\gx}{g}
\newcommand{\gxt}{\gx^\twinsupscript}
% \newcommand{\hx}{h}
\newcommand{\hxt}{\hx^\twinsupscript}

% \newcommand{\N}{T}
\newcommand{\tx}{t}
\newcommand{\B}{B}
\newcommand{\N}{N}
\newcommand{\T}{T}

\newcommand{\fx}{f}

\newcommand{\lo}[1]{#1_{\mathrm{lo}}}
\newcommand{\up}[1]{#1_{\mathrm{up}}}
\newcommand{\loup}[1]{{#1_{\mathrm{lo} \cap \mathrm{up}}}}

\newcommand{\Y}{Y}
\newcommand{\Yt}{\widehat{Y}}
\newcommand{\ylo}{\lo{y}}
\newcommand{\yup}{\up{y}}
\newcommand{\ytlo}{\lo{y^\twinsupscript}}
\newcommand{\ytup}{\up{y^\twinsupscript}}
\newcommand{\Ylo}{\lo{\Y}}
\newcommand{\Yup}{\up{\Y}}

\newcommand{\Q}{Q}
\newcommand{\Qt}{\widehat{\Q}}
\newcommand{\Qlo}{\lo{\Q}}
\newcommand{\Qup}{\up{\Q}}

\newcommand{\Qlohat}{\lo{\widehat{\Q}}}
\newcommand{\Quphat}{\up{\widehat{\Q}}}

\newcommand{\qlo}[1]{\lo{R}^{#1}}
\newcommand{\qup}[1]{\up{R}^{#1}}
\newcommand{\qt}[1]{\widehat{R}^{#1}}

\newcommand{\mybar}[1]{\makebox[0pt]{$\phantom{#1}\overline{\phantom{#1}}$}#1}
\newcommand{\Ylobar}{\lo{\mybar{\Y}}}
\newcommand{\Yupbar}{\up{\mybar{\Y}}}

\newcommand{\YClmean}{{\lo{\mu}}}
\newcommand{\YCumean}{{\up{\mu}}}
\newcommand{\Ytmean}{\widehat{\mu}}
\newcommand{\CIlen}{\Delta}
\newcommand{\CIlolen}{\lo{\Delta}}
\newcommand{\CIuplen}{\up{\Delta}}

\newcommand{\Hyp}{\mathcal{H}}
\newcommand{\Hlo}{\lo{\Hyp}}
\newcommand{\plo}{p_{\textup{lo}}}
\newcommand{\Hup}{\up{\Hyp}}
\newcommand{\pup}{p_{\textup{up}}}
\newcommand{\Hloup}{\loup{\Hyp}}
\newcommand{\test}{\delta}
\newcommand{\testlo}{\lo{\test}}
\newcommand{\testup}{\up{\test}}
\newcommand{\testloup}{\loup{\test}}
\newcommand{\reject}{\mathsf{reject}}
\newcommand{\accept}{\mathsf{retain}}

\newcommand{\D}{\mathcal{D}}
\newcommand{\Dt}{\widehat{\D}}

\newcommand{\QR}{R}
\newcommand{\QtR}{\widehat{R}}

\newcommand{\nsplitx}{n_{\mathrm{split}}}

\newcommand{\twinfunction}{h}
\newcommand{\twinnoise}{U}
\newcommand{\ux}{u}

\newcommand{\rob}[1]{{\color{blue}$\langle$ \textsc{Rob:} {\it #1} $\rangle$}}
% \newcommand{\Rob}[1]{{\color{orange}$\langle$ \textsc{Rob:} {\it #1} $\rangle$}}
% \newcommand{\faaiz}[1]{}
% \newcommand{\todiscuss}[1]{{\color{green}$\langle$ \textsc{To discuss:} {\it #1} $\rangle$}}

\newcommand{\Z}{Z}
\newcommand{\Zt}{\widehat{Z}}

\newcommand{\clip}{\mathrm{clip}}

% \newcommand{\Rob}[1]{}
% \newcommand{\rob}[1]{}
% \newcommand{\todiscuss}[1]{}
% \newcommand{\faaiz}[1]{}

\newcommand{\Pobs}{{P_\mathrm{obs}}}
\newcommand{\Pobstilde}{{\tilde{P}_\mathrm{obs}}}


\newcommand{\AppendixName}{Appendix\xspace}
\usepackage{tabularx}

% \newtheorem{manualtheoreminner}{Theorem}

% \newenvironment{manualtheorem}[1]{%
%   \renewcommand\themanualtheoreminner{#1}%
%   \manualtheoreminner
% }{\endmanualtheoreminner}


% \newenvironment{manualtheorem}[1]{%
%   \begin{mainresult}    
%   \renewcommand\themanualtheoreminner{#1}%
%   \manualtheoreminner
% }{
%     \endmanualtheoreminner 
%     \end{mainresult}
% }

% \newenvironment{manualtheorem}[1]{%
%   \renewcommand\themanualtheoreminner{#1}%
%   \manualtheoreminner
% }{
%     \endmanualtheoreminner 
%     \end{mainresult}
% }
\newenvironment{manualtheorem}[1]
 {\begin{mainresultwithtitle}[title={Theorem #1}]\noindent}
 {\end{mainresultwithtitle}}

% \newtheorem{manualpropositioninner}{Proposition}
% \newenvironment{manualproposition}[1]{%
%   \renewcommand\themanualpropositioninner{#1}%
%   \manualpropositioninner
% }{\endmanualpropositioninner}


% \newtheorem{manualpropositioninner}{Proposition}
% \newenvironment{manualproposition}[1]{%
% \begin{mainresult}
%   \renewcommand\themanualpropositioninner{#1}%
%   \manualpropositioninner
% }{
%     \endmanualpropositioninner
%     \end{mainresult}
% }
\newenvironment{manualproposition}[1]
 {\begin{mainresultwithtitle}[title={Proposition #1}]\noindent}
 {\end{mainresultwithtitle}}


%%%%%%%%%%%%%%%%%%%%%%%%% MR %%%%%%%%%%%%%%%%%%%%

\newcommand{\tar}[0]{\pi^*}
\newcommand{\beh}[0]{\pi^b}
\newcommand{\hatbeh}[0]{\widehat{\pi}^b}
\newcommand{\ptar}[0]{p_{\pi^*}}
\newcommand{\Etar}[0]{\mathbb{E}_{\pi^*}}
\newcommand{\Etarred}[0]{\mathbb{E}_{\textcolor{red}{\pi^*}}}
\newcommand{\Ebeh}[0]{\mathbb{E}_{\pi^b}}
\newcommand{\Ebehblue}[0]{\mathbb{E}_{\textcolor{blue}{\pi^b}}}
\newcommand{\Vtar}[0]{\textup{Var}_{\pi^*}}
\newcommand{\Vbeh}[0]{\textup{Var}_{\pi^b}}
\newcommand{\V}[0]{\textup{Var}}

\newcommand{\pbeh}[0]{p_{\pi^b}}
\newcommand{\thetacdr}[0]{\hat{\theta}_{\textup{CDR}}}
\newcommand{\thetadr}[0]{\hat{\theta}_{\textup{DR}}}
\newcommand{\thetamr}[0]{\hat{\theta}_{\textup{MR}}}
\newcommand{\thetaipw}[0]{\hat{\theta}_{\textup{IPW}}}
\newcommand{\thetaswitch}[0]{\hat{\theta}_{\textup{SwitchDR}}}
\newcommand{\thetadros}[0]{\hat{\theta}_{\textup{DRos}}}
\newcommand{\approxipw}[0]{\tilde{\theta}_{\textup{IPW}}}
\newcommand{\approxmr}[0]{\tilde{\theta}_{\textup{MR}}}
\newcommand{\thetagmdr}[0]{\tilde{\theta}_{\textup{DM-DR}}}
\newcommand{\ipw}[0]{\textup{IPW}}
\newcommand{\cdr}[0]{\textup{CDR}}
\newcommand{\mr}[0]{\textup{MR}}
\newcommand{\dr}[0]{\textup{DR}}
\newcommand{\ate}[0]{\textup{ATE}}
\newcommand{\var}[0]{\textup{Var}}
\newcommand{\Yspace}[0]{\mathcal{Y}}

\newcommand{\iidsim}[0]{\overset{\textup{i.i.d.}}{\sim}}


\newcommand{\Dtr}[0]{\mathcal{D}_{\textup{tr}}}
\newcommand{\Dev}[0]{\mathcal{D}_{\textup{ev}}}
\newcommand{\gt}[0]{{\textup{gt}}}

\newcommand{\ateipw}{\widehat{\ate}_{\ipw}}
\newcommand{\atemr}{\widehat{\ate}_{\mr}}
\newcommand{\atedr}{\widehat{\ate}_{\dr}}
\newcommand{\thetasnmr}{\theta_{\textup{SNMR}}}
\newcommand{\tr}{{\textup{tr}}}
\newcommand{\myparagraph}[1]{\paragraph{#1}}
\newcommand{\flag}[1]{#1}
\usepackage{wrapfig}
\usepackage{tikz}
\usepackage{ifthen}
\usepackage{rotating}

% Define a boolean flag
\newboolean{compilePapers}
\newboolean{compileAppendices}
% Set the flag to true or false
\setboolean{compilePapers}{true}% or \setboolean{myflag}{false}

\setboolean{compileAppendices}{true}
